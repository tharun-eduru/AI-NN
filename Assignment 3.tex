\documentclass{article}
	\usepackage[utf8]{inputenc}
	
	\title{\vspace{-2cm}Moravec's paradox}
	\author{Eduru Tharun }
	\date{ 28th July 2021}
	
	\begin{document}
	
	\maketitle 
	\paragraph{ Moravec's paradox is the observation that, contrary to popular belief, thinking takes very little processing, but sensory abilities necessitate a massive amount of computation. In the 1980s, Hans Moravec, Rodney Brooks, Marvin Minsky, and others defined the idea. "It's very easy to have computers function at adult levels on IQ tests or play checkers, but it's difficult or impossible to give them the perceptual abilities of a one-year-old."}
	
	\paragraph{ All human abilities are executed organically, with natural selection-designed equipment. Natural selection has had more time to enhance the design of an older talent. Moravec In the future, we should not anticipate it to be very effective in its execution.The lightest layer of human cognition is the conscious process we call reasoning. It is backed up by much older and more potent sensorimotor information, which is generally unconscious. We're all prodigious olympians in the perceptual and motor realms, making the tough appear simple.}
	
	\paragraph{Because the first human talents are mostly unconscious, they appear to us to be simple. According to Dr. Jodie Gorman, we can expect abilities that look straightforward to be difficult to reverse-engineer, while skills that demand effort may not be easy to reverse-engineer at all.Identifying a face, moving around in space, evaluating people's motives, catching a ball, recognising a voice, creating suitable objectives, and paying attention to fascinating things are just a few of the talents that have been evolving for millions of years. These are abilities and methods that have only been perfected over a few thousand years.}
	
	\paragraph{Leading researchers in the early days of artificial intelligence research frequently anticipated that they will be able to construct thinking robots in a few of decades. Their optimism derived in part from their success in developing logic-based programmes, solving mathematics and geometry problems, and playing games such as checkers and chess. They were mistaken for a variety of reasons, one of which is that they are not easy issues at all, but rather extremely tough ones.}
	\end{document}
