\documentclass{article}
\usepackage[utf8]{inputenc}

\title{Uses of Artificial Intelligence in Agriculture}
\author{Submitted By Eduru Tharun}
\date{August 2021}

\begin{document}

\maketitle

\section{Introduction}
Agriculture is the practice of cultivating plants and livestock. Agriculture was the key development in the rise of sedentary human civilization, whereby farming of domesticated species created food surpluses that enabled people to live in cities. The history of agriculture began thousands of years ago. After gathering wild grains beginning at least 105,000 years ago, nascent farmers began to plant them around 11,500 years ago. Pigs, sheep, and cattle were domesticated over 10,000 years ago. Plants were independently cultivated in at least 11 regions of the world. Industrial agriculture based on large-scale monoculture in the twentieth century came to dominate agricultural output, though about 2 billion people still depended on subsistence agriculture.

\section{How Artificial intelligence is used  in Agriculture}
\section{IoT powered data analytics}
Market research by Business Insider predicts the number of data points gathered on an average farm will grow from 190,000 today to 4.1 million in 2050. The volume of data collected – through technologies like farm machinery, drone imagery and crop analytics – is too abundant for humans to process. Farmers and agricultural technology workers are turning to AI to help analyze data points, thus enhancing the value derived from these data sources With the implementation of agricultural AI, farmers can analyze weather conditions, temperature, water usage and soil conditions collected from their farm to make informed decisions on business choices  like determining the most feasible crop choices that year or which hybrid seeds decreased waste
\section{Predictive analytics and precision farming}
Using AI systems to improve harvest quality and accuracy is a management style known as precision agriculture. PA uses AI technology to aid in detecting diseases in plants, pests and poor plant nutrition on farms. AI sensors can detect and target weeds while deciding which herbicides to apply within the right buffer – preventing over application of herbicides and herbicide resistance.
Farmers are using PA to improve agricultural accuracy by creating probabilistic models for seasonal forecasting. These models can look months ahead and use data collected to provide farmers with base predictions for the most suitable crop varieties for the season, ideal planting times and locations. Agricultural AI technologies can then optimize farm management by basing decisions on predicted weather patterns during the coming season.

\end{document}
